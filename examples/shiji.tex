\documentclass[四库全书]{ltc-guji}
\title{欽定四庫全書}
\chapter{史記\\卷一}

\begin{document}
\begin{正文}

欽定四庫全書

\begin{列表}
    \item 史記卷一
    \item 五帝本紀第一
\end{列表}

\begin{段落}[indent=3]
\夹注{集解裴駰曰凡是徐氏義稱徐姓名以别之餘者悉是駰註解并集衆家義索隠紀者記也本其事而記之故曰本紀又紀理也絲縷有紀而帝王書稱紀者言為後代綱紀也正義鄭玄注中候勅省圖云徳合五帝坐星者稱帝又坤靈圖云徳配天地在正不在私曰帝}
\end{段落}

黄帝者\夹注{集解徐廣曰號有熊索隠按有土徳之瑞土色黄故稱黄帝猶神農火徳王而稱炎帝然也此以黄帝為五帝之首蓋依大戴禮五帝徳又譙周宋均亦以為然而孔安國皇甫謐帝王代紀及孫氏註系本並以伏犧神農黄帝為三皇少昊髙陽髙辛唐虞為五帝}少典之子\夹注{集解譙周曰有熊國君少典之子也皇甫謐曰有熊今河南新鄭是也索隠少典者諸侯國號非人名也又按國語云少典娶有蟜氏女而生炎帝然則炎帝亦少典之子}姓公孫名曰軒轅\夹注{索隠按皇甫謐云黄帝生於壽丘長於姬水因以為姓居軒轅之丘因以為名又以為號是本姓公孫長居姬水因改姓姬}生而神靈弱而能言\夹注{索隠弱謂幼弱时也盖未合能言之时而黄帝即言所以为神异也}幼而徇齊\夹注{集解徐廣曰墨子曰年踰十五}

\end{正文}
\end{document}
