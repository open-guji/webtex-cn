\documentclass[红楼梦甲戌本]{ltc-guji}
\title{脂硯齋重評石頭記}
\chapter{第一回}

\begin{document}
\begin{正文}

脂硯齋重評石頭記

\begin{列表}
    \item 甲戌本
    \item 第一回
\end{列表}

\begin{段落}[indent=2]
此開卷第一回也作者自云曾歷過一番夢幻之後故將真事隱去而借通靈說此\夹注{石頭記}一段故事也
\end{段落}

列位看官你道此書從何而來說起根由雖近荒唐細按則深有趣味

\侧批{此是第一首標題詩}

待在下將此來歷註明方使閱者了然不惑\\
當日地陷東南\夹注{洪荒之世女媧氏煉石補天之時}這東南一隅有處曰姑蘇\夹注{有城曰閶門者最是紅塵中一二等富貴風流之地}

\侧批{妙}

這閶門外有個十里街街內有個仁清巷巷內有個古廟因地方窄狹人皆呼作葫蘆廟\\
廟旁住著一家鄉宦姓甄名費字士隱\夹注{真事隱也}嫡妻封氏情性賢淑深明禮義

\圈点{家中雖不甚富}然本地也推他為望族了

\专名号{甄士隱}禀性恬淡不以功名為念每日只以觀花修竹酌酒吟詩為樂\\
倒是神仙一流人品

\书名号{石頭記}是中國四大名著之一

\反白{卷一}

\批注[x=3cm, y=1cm]{甲戌眉批:開卷一篇立意真打破歷來小說窠臼}

\end{正文}
\end{document}
