\documentclass[四库全书文渊阁简明目录]{ltc-guji}

\usepackage{enumitem} % For better list control if needed
\usepackage{tikz}
% \禁用分页裁剪
\句读模式
\setmainfont{TW-Kai}


\title{欽定四庫全書簡明目錄}

\chapter{經部 易類\\卷一}

\begin{document}
\begin{正文}
欽定四庫全書簡明目錄
\条目[1]{卷一}
\条目[2]{經部一}
\条目[3]{易類}
\条目[1]{卷二}
\条目[2]{經部二}
\条目[3]{書類}
\条目[2]{經部三}
\条目[3]{詩類}
\条目[2]{經部四}
\条目[3]{禮類}
\条目[1]{卷三}
\条目[2]{經部五}
\条目[3]{春秋類}
\条目[2]{經部六}
\条目[3]{孝經類}


欽定四庫全書簡明目錄卷一

\条目[1]{經部一}
\条目[2]{易類}
    
《子夏易傳》十一卷

\注{舊本題卜子夏撰,實後人輾轉依托,非其原書。然唐、宋以來,流傳已久,今仍錄冠易類之首。凡托名之書,仍從其所托之時代,《漢書·藝文志》例也。}

\按{謹按:唐徐堅《初學記》以太宗御制升列歷代之前,蓋尊尊之大義。焦竑《國史經籍志》,朱彞尊《經義考》並踵前規。臣等編摩《四庫》,初亦恭錄
\单抬《御定易經通注》
\单抬《御纂周易折中》
\平抬《御纂周易述義》,弁冕諸經。仰蒙\\
指示,命冠於}

\注{國朝著述之首,俾尊卑有序,而時代不淆。\平抬 聖度謙沖,酌中立憲,實為千古之大公。謹恪遵\平抬 彞訓,仍託始於《子夏易傳》。並發凡於此,著《四庫》之通例}
\按{焉。}

《周易鄭康成注》一卷

\注{漢鄭玄撰。原本散佚,此本乃宋末王應麟采諸書所引,裒合而成。}

\按{前代佚書而後人重編者,如有所竄改,則從重編之時代。如全輯舊文者,則仍從原書之時代。故此書雖宋人所輯,而列於漢代之中。後皆仿此。}

《新本鄭氏周易》三卷

\注{漢鄭玄撰,\\
國朝惠棟編,因王應麟之本,採摭未備,又不註其所出,因重為補正,凡增入九十二條。又據鄭氏《周禮》《禮記》註,作十二月爻辰及爻辰直二十八宿圖,以闡明漢學。}

《陸氏易解》一卷

\注{吳陸績撰。原本散佚,明姚士粦採陸氏《經典釋文》、李氏《周易集解》及績《京氏易傳註》輯為此本,凡一百五十條。}

《周易註》十卷

\注{魏王弼註。其《繫辭》以下,則韓康伯註也。漢氏《易》學皆明象數,至弼始黜象數而言義理,足以糾讖緯之失,而語涉老、莊,亦開後來玄虛之漸。}

《周易正義》十卷

\注{唐孔穎達撰。穎達諸經正義,皆元元本本,引據詳明,惟《周易》罕徵典籍。蓋所疏者,王、韓之註;而王、韓皆掃棄舊聞,自標新解,故不能以漢儒古義與之證明,非其考訂之疏也。}

《周易集解》十七卷

\注{唐李鼎祚撰。凡採《子夏易傳》以下三十五家之說。鼎祚自序稱:「刊輔嗣之野文,補康成之逸象。」蓋發明漢學者也。}

《周易口訣義》六卷

\注{唐史徵撰。大旨與李鼎祚書相類,而與李書互有詳略,且多李書所未載。世無傳本,今從《永樂大典》錄出,為罕觀之祕笈。}

《周易舉正》三卷

\注{舊本題唐郭京撰。自序稱:「得晉王弼、韓康伯手寫《周易》真本,刊正今本訛脫一百三十五條。」朱子《本義》亦採用其說。然《唐書·藝文志》不著錄,至北宋始出,晁公武等多疑其依托。}

《易數鉤隱圖》三卷,附《遺論九事》一卷

\注{宋劉牧撰。其說出於陳摶與邵子先天之學,異派同源,惟以九數為《河圖》,十數為《洛書》,與邵子異。宋人《易》數以此書為首。其「遺論九事」,皆奇偶陰陽之說,先儒之所未言者也。}

《周易口義》十二卷

\注{宋倪天隱述其師胡瑗之說,故曰「口義」。大旨主闡明義理,程子之《易源》從此出。}

《溫公易說》六卷

\注{宋司馬光撰。其書宋代有兩本,皆已散佚,此本為《永樂大典》所載,即《朱子語錄》所謂「北方互市之完本」也。大旨在闡明人事,不主空虛玄妙之說。}

《橫渠易說》二卷

\注{宋張載撰。文頗簡略,蓋無可發揮新義者,即不橫生枝節,強為敷衍,猶有先儒篤實之遺。間有引用《老》《莊》語者,蓋借以旁證,非祖其虛無之談。}

《東坡易傳》九卷

\注{宋蘇軾撰。其大體近於王弼,然弼說惟暢玄風,軾說多切人事,實不相同。朱子作《雜學辨》,嘗摘駁其中十九條,然不害其全書也。}

《伊川易傳》四卷

\注{宋程頤撰,其門人楊時校正。經文用王弼之本,惟解《上下經》彖象及文言亦與弼同。大旨黜數而崇理,與邵子各明一義。}

《易學辨惑》一卷

\注{宋邵伯溫撰。伯溫,邵子之子也。以同時鄭夬詭稱得邵子之傳,所作說《易》諸書,支離破碎,多乖經義,因作此書以辨其誣。原本久佚,今從《永樂大典》錄出。}

《了翁易說》一卷

\注{宋陳瓘撰。瓘之學出於邵氏,又常質於劉安世。故其說,理數兼推。陳振孫《書錄解題》頗病其詞旨深晦,然晁公武《讀書志》則稱其數之多驗云。}

《吳園易解》九卷

\注{宋張根撰。不主漢儒象數之說,亦不主宋代河洛之學,詮釋經文,頗為簡切。末附《泰卦論》一篇,深著滿盈之戒,蓋作於徽宗之世,有為而發也。}

《周易新講義》十卷

\注{宋耿南仲撰。南仲當欽宗之時,力主割地,為史傳所譏。然是書因象詮理,隨事示戒,乃頗有可取。自序謂「《易》主於無咎,無咎在於善補過」,而大旨歸於無拂天道。}

《紫巖易傳》十卷

\注{宋張浚撰。其說發揮《易》理,頗為醇正明白。惟末卷《雜論》,以九數為《河圖》,主劉牧之說,與朱、蔡異,然亦無關宏旨也。}

《讀易詳說》十卷

\注{宋李光撰。舊本散佚,今從《永樂大典》錄出。書中於卦爻之辭,皆引証史事,以君臣立論,或不免有所牽合。然意存法戒,究勝空談,援古事以證爻象,始自鄭玄;若全經皆證以史,則光書其始也。}

《易小傳》六卷

\注{宋沈該撰。其書以正體發明爻象之旨,以變體擬議變動之意。其占則全用《左傳》所載筮例。在南宋人《易》說之中,為獨存古法。}

《漢上易集傳》十一卷,《卦圖》三卷,《叢說》一卷

\注{宋朱震撰。其書以數為宗,闡陳、邵河洛先天之學,而兼採漢以來卦變、互體、伏卦、反卦諸說。頗為蕪雜。然得失互陳,存之亦可資參考。}

《周易窺餘》十五卷

\注{宋鄭剛中撰。其書以《伊川易傳》主理,《漢上易傳》主數,參取兩家,發所未盡,故名曰「窺餘」。大旨兼採漢學,而增以新義,不甚拘守成說,然往往愜當於理。原本久佚,今從《永樂大典》錄出。}

《易璇璣》三卷

\注{宋吳沆撰。自序謂:上卷明天理之自然,中卷講人事之修,下卷備傳疏之失。凡論二十七篇。其曰「璇璣」者,取《易略例》,處「璇璣以觀大運」語也。}

《易變體義》十二卷

\注{宋都潔撰。其書專明變體,即《左傳》所載諸占、某卦之某卦者是也。原本久佚,今從《永樂大典》錄出。}

《周易經傳集解》三十六卷

\注{宋林慄撰。其說每卦必兼言互體約象複卦。嘗與朱子論太極、兩儀、四象、八卦不合,至於互劾。故講學家最惡其書,幾於不傳。然《易》道廣大,各明一義,不必定執門戶之見也。}

《易原》八卷

\注{宋程大昌撰。其書推闡數學,故謂之《易原》。於京、焦卦氣,馬、鄭爻辰,以及邵子、張行成諸說,皆一一掊擊,務申己說,未免失之好辨。而根據《繫辭》,於《易》義亦有所發明,非盡鑿空立異。舊本久佚,今從《永樂大典》錄出。}

《周易古占法》一卷,《古周易章句外編》一卷

\注{宋程迥撰。舊本傳寫,混二書為一。今考《宋史》釐正。前卷論占法,後卷雜說《易》義及占驗。其說用邵子加一倍法,據《繫辭》《說卦》,發明其義,用逆數以尚占知來。}

《原本周易本義》十二卷

\注{宋朱熹撰。坊刻此書,皆改從《程傳》之次第。此本以《經》為二卷,《十翼》為十卷,猶朱子之原本也。}

《別本周易本義》四卷

\注{{{{謹案:《總目》此部不存。}}}明成矩撰。割裂朱子《易本義》,以附《程傳》之後,始元董楷,而明永樂《大全》因之。後場屋專用《本義》,而《大全》以官本不敢改。矩因刊為是本,以調停其間,相沿日久,今亦姑與原本存焉。}

《郭氏傳家易說》十一卷

\注{宋郭雍撰,以述其父忠孝「兼山易解」之旨,故名曰「傳家」。自序謂:《易》之為書,其道其詞,皆由象出,未有忘象而知《易》者。大旨以觀象為主,然剖析義理,猶守程門之規範,蓋其父忠孝,即程子之門人也。}

《周易義海撮要》十二卷

\注{宋李衡刪定。初,熙寧中,蜀人房審權病《易》說多岐,摘取專明人事者,由鄭玄迄王安石凡一百家,共為一百卷,名《周易義海》。衡病其蕪雜重複,乃刪掇精要,以成此書,故名曰「撮要」。}

《南軒易說》三卷

\注{宋張栻撰。原本十一卷。此本出自曹溶家,《上下經》全佚,惟存繫辭。然《繫辭傳》托始於「天一地二」一章,亦非完本。蓋元人刊本,以程子《易傳》缺《繫辭》,割栻書補之,後又佚其前半也。}

《復齋易說》六卷

\注{宋趙彥肅撰。其說推尋卦畫,即象數以求其理。朱子《語錄》,頗病其取義太密;然研索於《易》中,究勝支離於易外也。}

《楊氏易傳》二十卷

\注{宋楊簡撰。簡為陸九淵之弟子,故其說《易》,略象數而談心性,多入於禪。錄存其書,見以佛理詁《易》,自斯人始,著經學別派之由也。}

《周易玩詞》十六卷

\注{宋項安世撰。前有自述,稱其學以伊川《易傳》為宗。然立說頗與伊川異,蓋伊川務闡義理,安世則兼言象數以補所遺,故與尺寸步趨者殊焉。}

《趙氏易說》四卷

\注{宋趙善譽撰。舊本二卷久佚,今從《永樂大典》錄出,釐為四卷。其書推畫卦命名之意,以貫通六爻之旨。於諸卦取義相似者,參互以盡其變,往往具有精理。}

《誠齋易傳》二十卷

\注{宋楊萬里撰。其書大旨本程氏,而參引史傳以證之,則與李光之書相同。講學家如吳澄、陳櫟、胡一桂等,皆不滿之。蓋門戶之見,不足據也。}

《大易粹言》十卷

\注{宋方聞一編。《宋史·藝文志》作曾穜者誤也。是書採二程子、張子、楊時、游酢、郭忠孝、郭雍七家之說,皆程氏之宗派,知其以洛學為主矣。}

《易圖說》三卷

\注{宋吳仁傑撰。其說以六十四正卦,伏羲所作;卦外六爻及六十四複卦,文王所作。又謂《序卦》為伏羲作,《雜卦》為文王作;今之爻辭,當為《系辭傳》;《系辭傳》,當為《說卦傳》。皆故為異說。宋人舊帙,姑存備一解云爾。}

《古周易》一卷

\注{宋呂祖謙編。自王弼以後,《周易》皆以傳附經。呂大防以下諸家互有考定,而小有異同。祖謙乃以《上下經》《十翼》各為一篇,復古本之舊。朱子《本義》,即用此本也。}

《易傳燈》四卷

\注{是書世無傳本,諸家書目皆不著錄,《永樂大典》收之,題曰「宋徐總幹撰」,亦不著其名。惟據原序,知為呂祖謙之門人耳。其以釋氏「傳燈」命名,頗為乖剌;參以五行家言,亦為駁雜。然其「八卦總論」十六篇,參互以求,頗能得《易》之類例。}

《易裨傳》二卷

\注{宋林至撰。上卷凡三篇,一曰法象,一曰極數,一曰觀變。下卷題曰外篇。}

《厚齋易學》五十二卷

\注{宋馮椅撰。舊本散佚,今從《永樂大典》錄出。從其自序,釐為《輯注》四卷、《輯傳》三十卷、《外傳》十八卷。《輯注》惟解彖象;《輯傳》則以彖象為經,而《十翼》為傳;《外傳》則以《十翼》為經。各附先儒之說,而斷以己意。}

《童溪易傳》三十卷

\注{宋王宗傳撰。其說力排象數,而不免涉於虛無。大旨與楊簡相類,二人同時,未知孰倡孰和也。}

《周易總義》二十卷

\注{宋易祓撰。祓本蘇師旦之黨人,不足道。然其說《易》,兼該理數,折中眾論。每卦先括為總論,復於六爻之下,詳為詮釋,於經義乃頗有發明。}

《西溪易說》十二卷

\注{宋李過撰。其書首為序說一卷,次詮釋經文,而不及《繫辭》以下。胡一桂譏其於經文多所竄亂,馮椅則稱其多所發明,蓋瑕瑜不掩之書也。}

《丙子學易編》一卷

\注{宋李心傳撰。書成於嘉定丙子,因以為名。所取惟王弼、張子、程子、郭雍、朱子五家之說,而以其父舜臣之說證之,亦間附以己意。原本十五卷,歲久散佚,此本乃宋末俞琰所節抄,略存梗概而已。}

《易通》六卷

\注{宋趙以夫撰,或以為莆田黃績所代作。趙汝騰至見彈章,莫能詳也。大旨以不易、變易二義參互,以明人事動靜之準。}

《周易經傳訓解》二卷

\注{宋蔡淵撰。原本四卷,今佚其二卷,惟存《上經》《下經》。其經文以大象置卦辭下,以彖傳置大象後,以小象置爻辭後,皆低一字,以別卦爻。與舊本小異。其訓釋則明義理者居多。}

《易象意言》一卷

\注{宋蔡淵撰。原本久佚,今從《永樂大典》錄出。淵,元定之子,而從學於朱子。故此書闡發名理,多從師說;兼言數學,則本其家傳。其兼用互體,則取裁古義,與講學家持論又殊。}

《周易要義》十卷

\注{宋魏了翁撰,其《九經要義》之一也。即孔穎達《周易正義》,刪繁舉要,以便循覽,體例頗為簡當。}

《東谷易翼傳》二卷

\注{宋鄭汝諧撰。所謂「翼傳」者,翼伊川《易傳》也。然於程子之說,亦時有異同,蓋糾正其失,補苴其闕,亦所以羽翼之,可謂無朋黨之私矣。}

《文公易說》二十三卷

\注{宋朱鑒編。鑑為朱子之長孫,是書裒輯朱子平日論《易》之語,見於《語錄》《文集》者,共為一編,以發明《本義》之旨。}

《易學啟蒙小傳》一卷

\注{宋稅與權撰。朱子作《易學啟蒙》,多發明邵子《先天圖》義,至於後天之易,則以為不得文王所以安排之意。與權研求邵子之說,知易有不易之八卦為乾,有互易之五十六卦為用,反複觀之,《上下經》皆十八卦,羲、文之《易》似異而同。因作此書,以補朱子之所遺。}

《周易輯聞》六卷,附《易雅》一卷、《筮宗》一卷

\注{宋趙汝楳撰。《周易輯聞》,但解上、下經,多所發揮,惟竄亂經文,是其一失。《易雅》,總釋名義,凡十八篇,如《爾雅》之釋詩,故名曰「雅」。《筮宗》凡三篇,其中推闡大衍之數,頗為明皙。}

《周易詳解》十六卷

\注{宋李杞撰。以《易》為有用之學,故名「用易」。自序甚明。焦竑《經籍志》作「周易詳解」者,誤也。原本二十卷,久已散佚,今從《永樂大典》錄出,編為十六卷。其書多證以史事,與李光、楊萬里書同,惟頗參以老、莊之說,不免駁雜。}

《淙山讀周易記》二十一卷

\注{宋方實孫撰。其說多主爻象,不涉虛無。其《易卦變合圖》,補朱子《啟蒙》所未備。}

《周易傳義附錄》十四卷

\注{宋董楷撰。以程子之《傳》、朱子之《本義》,合為一書。又博採程、朱之說,附錄其下,使互相發明。惟割裂《本義》,以附《程傳》,自楷此書始。舊傳始於胡廣等修《周易大全》者,非也。}

《易學啟蒙通釋》二卷

\注{宋胡方平撰。是書發明朱子《易學啟蒙》之義,所採諸說,蔡淵等六家,皆朱子之門人。蔡模即淵之子,徐幾、翁詠,又皆淵之門人。所謂一家之學也。}

《三易備遺》十卷

\注{宋朱元升撰。首為《河圖》《洛書》一卷,祖劉牧之說。次《連山》三卷,以卦位配《夏時》之節氣。次《歸藏》三卷,以干支納音配卦爻。次《周易》三卷,皆發反對互體之旨。}

《周易集說》四十卷

\注{宋俞琰撰。琰初裒諸家《易》說,為《大易會要》一百三十卷。後乃掇其精華,以成是書。初惟主程、朱之說,後乃研索經文,浚發新義,自為一家之言。}

《讀易舉要》四卷

\注{宋俞琰撰。琰所著說易之書,凡十一種,今多散佚,此書乃從《永樂大典》錄出者也。琰說《易》多主朱子,而此書論剛柔往來,不主朱子卦變之說。其易圖多主邵子,而此書論元亨利貞,不主起數於四之說。亦可謂不苟異、不苟同矣。}

《易象義》十六卷

\注{宋丁易東撰。是書因象以明義,故曰「象義」。其取象之例,凡十有二,大抵以李鼎祚、朱震二家為宗。而卦變則取朱子,變卦則取都潔、沈該,筮占則取朱子、蔡淵、馮椅,亦不偏主於二家。}

《易圖通變》五卷,《易筮通變》三卷

\注{宋雷思齊撰。其《易圖通變》,以八卦配《河圖》,天一至地八,而以五十為虛數,與先儒之說頗異。其《易筮通變》分五篇,亦多自出新意,蓋奇偶相生,變化不窮,隨意錯綜,無不可以成理也。}

《讀易私言》一卷

\注{元許衡撰。是書論六爻之德位,大旨多發明《繫辭傳》同功異位、柔危剛勝之義。其謂各卦畫之居六位者,吉凶悔吝,視乎其時,而歸於正而得中;又彖傳當位、不當位得中、行中之義也。}

《易本義附錄纂注》十五卷

\注{元胡一桂撰。是書以朱子《本義》為宗,取朱子《文集》《語錄》之說《易》者附之,謂之「附錄」。又纂諸儒之說不悖於《本義》者,謂之「纂注」。蓋宋末、元初朱子之學盛行,儒者惟守一先生之言矣。}

《易學啟蒙翼傳》四卷

\注{元胡一桂撰。一桂之父方平嘗作《易學啟蒙通釋》。一桂更推闡辨別之,故曰「翼傳」。凡為內篇者三,皆發朱子占筮圖書之說;為外篇者一,皆雜論《易》學之支流。}

《易纂言》十卷

\注{元吳澄撰。澄於諸經多臆為竄亂,惟此經所改,大抵依據先儒,較為有本。其注釋經義,亦詞簡而理明。}

《易纂言外翼》八卷

\注{元吳澄撰。澄所作《易纂言》,義例散見各卦,不相統貫;卷首所列卦圖,亦粗具梗概,乃復作此書暢明之。凡十二篇,原本久佚,今從《永樂大典》錄出,尚缺其卦變、變卦、互卦三篇。易流、易原二篇,亦缺其半。然大旨亦可睹矣。}

《易源奧義》一卷,《周易原旨》六卷

\注{元寶巴撰。案:「寶巴」原本作「保八」,今改正。是書原名「易體用」,分為三種,今佚其「周易尚占」三卷,僅存其二,大旨皆祖述程、朱。}

《周易程朱傳義折衷》三十三卷

\注{元趙采撰。是書節錄程子《易傳》、朱子《本義》之文,益以《語錄》諸書,而各以己說附於後。所註惟上、下經,或以程子未注《系辭》以下故也。大旨雖宗宋學,而於象數變互尚頗存古義,所謂「折衷」者,殆在是歟。}

《周易衍義》十六卷

\注{元胡震撰,其子廣大,《四庫總目》作光大,凡兩見。續成之。於經文次序,臆為顛倒,殊嫌乖剌。其雜引史事,亦稍傷泛濫,然持論尚不失為醇正。}

《易學濫觴》一卷

\注{元黃澤撰。其說《易》以明象為本,其明象以《序卦》為本,其占法則以《左傳》為主。大旨不取王弼之玄虛,亦不取漢儒之附會,故折中以酌其平。其歷陳《易》學不能復古者十三事,亦具有根據。}

《大易緝說》十卷

\注{元主申子撰。前二卷論數學,於陳、邵諸家之說,概斥其有誤;其所取者,自《河圖》《洛書》外,惟伏羲、文王、周公、孔子、周子五人,未免好為高論。然自三卷以下,詮釋經文,仍以辭變象占、乘承比應為說,又未嘗不平正切實。}

《周易本義通釋》十二卷

\注{元胡炳文撰。大旨以朱子《本義》為宗,而參以眾說。原本殘缺,惟上、下經僅存。其《十翼》,乃炳文九世孫珙、玠雜採他書所引炳文之說,以補之也。}

《周易本義集成》十二卷

\注{元熊良輔撰。是書成於延佑復科舉之後。元制,程試《易》用程氏、朱氏,而亦兼用古注疏。故是書雖以羽翼《本義》為主,而亦不盡墨守《本義》焉。}

《大易象數鉤深圖》三卷

\注{元張理撰。其書皆即陳摶、邵子之說,推廣成圖。朱子所謂「易外別傳」者是也。}

《學易記》九卷

\注{元李簡撰。是書仿李鼎祚《集解》,房審權《義海》之例,採《子夏易傳》以下六十四家之說,亦間附以己意。諸家之書,今十不存一,其佚文惟賴此書以存。}

《周易集傳》八卷

\注{元龍仁夫撰。每卦之下,各分象變辭占,雖大旨根據程、朱,而於卦象、爻象反複推闡,頗能自抒心得。故《元史》稱其「發前儒所未發」。原書十八卷,今佚十卷,然上、下經,彖、象傳皆尚完具,未可以殘缺廢也。}

《讀易考原》一卷

\注{元蕭漢中撰。是書凡三篇:一論分卦,一論合卦,一論序卦,不敢顯攻《序卦傳》,而亦不用序卦之說。大旨雖亦出陳、邵,而推衍頗有精理,尚不失為依經立義。}

《易精蘊大義》十二卷

\注{元解蒙撰。原本散佚,今從《永樂大典》錄出。其例於彖爻之下,採輯舊說,末乃發明以己意。雖為程試而作,然薈萃群言,頗有持擇,所自注亦皆簡明。}

《易學變通》六卷

\注{元曾貫撰。原本散佚,今從《永樂大典》錄出。其例每篇統論一卦六爻之義,又舉他卦辭義之相近者,參互以求異同之故,頗為融貫。其兼取互體,亦能存古義。}

《周易會通》十四卷

\注{元董真卿撰。真卿受業於胡一桂,此書即因一桂《纂疏》而廣之。然一桂堅持門戶,真卿則謂諸家之《易》,途雖殊而歸則同,故兼採象數、義理兩家,以持其平。即蘇軾、林栗之書,朱子所不取者,亦不掩其長,則所見視其師為廣矣。}

《周易圖說》二卷

\注{元錢義方撰。是書凡二十七圖,大抵衍陳、邵之緒餘。然如謂「《繫辭》兼言《河圖》《洛書》,乃言其理相通,非據《洛書》以作《易》」;又謂「陳摶因《易》而演《圖》,非伏羲據《圖》以畫卦」。皆篤論也。}

《周易爻變義蘊》四卷

\注{元陳應潤撰。大旨謂王弼所注,乃老、莊虛渺之談;陳摶所圖,乃《參同契》爐火之術,均非《易》之本旨。又謂:周子《太極圖》,別自一家之說,不可以釋《易》。皆能不域於門戶。所注惟六十四卦,其曰「爻變」,即衍《左傳》「某卦之某卦」之古義;其謂一卦可變六十四卦,亦焦、京舊法也。}

《周易參義》十二卷

\注{元梁寅撰。其說皆即日用常行之事,以示進退得失之機,頗為平易近人,勝於諸家之。}

《周易文詮》四卷

\注{元趙汸撰。汸於《易》學,不及《春秋》之深邃;然原本宋儒詮釋義理,於進退存亡之故,吉凶悔吝之理,推闡頗明。與梁寅書,皆切於人事者也。}

《周易傳義大全》二十四卷

\注{明永樂中翰林院學士胡廣等奉敕撰。其書魯莽而成,僅割裂董楷、董真卿、胡一桂、胡炳文四家之書,餖飣成編。以其為一代取士之制,故錄之以見經學盛衰之由焉。}

《易經蒙引》十二卷

\注{明蔡清撰。清篤信朱子之學,故是書體例,以《本義》與經文並書,但每條之首,加一圈以示別。然其立說,乃或與《本義》異同,蓋研索者深,故一一明其得失。猶陸游謂朱子尊程子,而說《易》乃與程子《傳》異同也。}

《讀易餘言》五卷

\注{明崔銑撰。凡《上下經卦略》二卷,《大象說》《繫辭輯說》《卦訓》各一卷。大旨以《程傳》為主,而兼採王弼、吳澄之說,不甚依附《本義》,論多切實。惟點竄《說卦》,而刪除《序卦》《文言》,未免勇於改經耳。}

《易學啟蒙意見》五卷

\注{明韓邦奇撰。凡五篇。前四篇皆推衍邵子、朱子之緒論。末一篇曰七占,凡六爻不變、六爻俱變及一爻變者,皆仍舊法,其二爻、三爻、四爻、五爻變者,則邦奇所立之新法也。}

《易經存疑》十二卷

\注{明林希元撰。是書繼蔡清《蒙引》而作,然小有異同。大旨為科舉而設,故謂漢學不可行於今。後來坊刻講章,此其濫觴。然明白篤實,終非後來講章所及也。}

《周易辨錄》四卷

\注{明楊爵撰。是書乃嘉靖乙巳,爵以建言下詔獄時所作。注惟六十四卦,經文但載卦辭,然注乃併解六爻、彖傳、象傳。其說多明人事,頗為剴切。}

《易象鈔》四卷

\注{明胡居仁撰。自序稱:讀《易》二十年,有所得輒抄積之。後二卷則皆與人論《易》之語,及自記所學,併為栝歌詞以舉其要。考萬歷乙酉,御史李頤請以居仁從祀,疏稱「所著《易傳》已散佚」,此本或後人所裒輯歟。}

《周易象旨決錄》七卷

\注{明熊過撰。據其自序,蓋因蔡清《蒙引》陳義而不及象,故作此書。名「決錄」者,猶言「定本」也。其說遠溯漢學,雖未必遽追梁、孟,然義必考古,終勝明人幻渺之談。}

《易象鉤解》四卷

\注{明陳士元撰。其說謂《易》以卜筮為用,卜筮以象為宗。雖或涉穿鑿,然犂然有當者居多。惟謂言象為京房之學,則殊舛誤。《京氏易》乃納甲飛伏之學,非以象為占也。}

《周易集注》十六卷

\注{明來知德撰。乃其空山獨處,研思二十九年而成。專取《繫辭》錯綜其數之說,以錯卦、綜卦論《易》象。其注皆先釋象義、字義及錯綜義,然後訓本卦、本爻正義。頗傷繁碎,而亦自成一家之學。}

《讀易紀聞》六卷

\注{明張獻翼撰。是書但隨筆札記,不載經文。其為人蕩檢踰閑,殆有狂疾,而說《易》乃篤實不支,多得聖人示戒之旨。蓋其早年力學,猶未放誕時作也。}

《葉八白易傳》十六卷

\注{明葉山撰。八白,其字也。惟釋六十四卦爻辭,大旨以誠齋《易傳》為宗,出入子史,佐以博辨。蓋借《易》以言人事,不必盡為經義之所有。然所言亦往往可昭法戒。}

《洗心齋讀易述》十七卷

\注{明潘士藻撰。每條皆先發己意,而採掇舊說列於後。焦竑序稱所採舊說,惟李氏《集解》、房氏《義海》二書。今觀所引,房書較多於李書。蓋李書主象,漢學之遺;房書主理,宋學之總。士藻所主者,宋學也。}

《周易像象管見》九卷

\注{明錢一本撰。一本所著《象抄》六卷,推衍陳摶之學,支離,殊無可觀。此書作於《象抄》之前,惟即卦爻以求象,即象以明人事,雖間有支蔓,而篤實近理者多。}

《周易札記》三卷

\注{明逯中立撰。是書不載經文,但以卦名、篇名為標識。採舊說者十之六,出新義者十之四。大旨以義理為主,而複、姤、中孚諸卦,亦兼用六日七分之說。}

《周易易簡說》三卷

\注{明高攀龍撰。其詮解《易》義,每條不過數言,故名曰「易簡」。亦頗闡明心學。然主於學《易》以檢心,非如楊簡、王宗傳輩,引《易》歸心,又引心歸禪也。}

《易義古象通》八卷

\注{明魏濬撰。前有《明象總論》八篇。大旨謂文周之《易》,即象著理;孔子之《易》,以理明象。因取漢、魏、晉、唐諸儒所論象義,取其近正者錄之。故名曰「易義古象通」,言即象以通義也。}

《周易像象述》五卷

\注{明吳桂森撰。乃踵其師錢一本《像象管見》而作,故以「述」為名。首列「像象金鍼」一篇,標舉大旨。卷中所注,皆一字一句,推尋義理,頗有新意。}

《易用》五卷

\注{明陳祖念撰。祖念,陳第子也,學不逮其父,而此書則勝其父《伏羲圖贊》遠甚。書中不載經文,但每卦、每章詳論其義,務以切於人事為主,故名曰「用」。每卦之末,總論取象之義,多取互體,蓋於漢學、宋學,無所偏主云。}

《易象正》十六卷

\注{明黃道周撰。於每卦六爻,皆即之卦以觀其變。蓋即《左傳》所載之古法。前列日次一卷,用漢人分爻直日之法。按文王卦序,以推世運。後二卷以《河圖》《洛書》自相乘除,推為三十五圖。則均《易》外之別傳矣。}

\按{此書及《三易洞璣》,皆《皇極經世》之支流。《三易洞璣》,全推衍於《易》外,故入之術數類。此及倪元璐《兒易》,有於《易》外者,猶有據經立義,發揮於《易》中者。且皆忠節之士,宜因人以重其書。故此二編仍著錄於經部,非通例也。}

《兒易內儀以》六卷,《兒易外儀》十五卷

\注{明倪元璐撰。名「兒易」者,據元璐自序,蓋取孩始之義。其「內儀以」專以大象釋經,以六十四卦大象皆有「以」字,故「以」為名也。《外儀》分六目,六目又各分子目,皆以《繫辭》中字義名篇,篇各有圖,大抵憂時傷亂,借《易》以抒其意,不必盡為經義之所有。}

《卦變考略》一卷

\注{明董守諭撰。以朱子《卦變圖》與《本義》自相矛盾,因考郎、京房、蜀才、虞翻諸家之說,推衍成圖,以存古義。}

《古周易訂詁》十六卷

\注{明何楷撰。前六卷以傳附經,用王弼本。七卷以下,則仍以《十翼》原文,存田、何之舊。其學雖博而不精,然取材宏富,詞必有據。漢、晉以來之古義,頗藉以見梗概。}

《周易玩詞困學記》十五卷

\注{明張次仲撰。自序謂不敢侈談象數,又雅不信讖緯之說,惟於語言文字間,求其有益於身心者,持論頗為篤實。其鏟除諸圖,亦具有廓清之力。}

《易經通註》九卷

\注{國朝大學士傅以漸等奉敕撰。順治十三年十二月,世祖章皇帝以永樂《易經大全》繁而可刪,華而寡要,因命以漸等刊其舛訛,補其缺漏,勒為是書。以順治十五年十月告成。}

《日講易經解義》十八卷

\注{康熙二十二年,大學士牛鈕等奉敕編。用宋代經筵講義之體,發揮要旨,疏通證明,不取莊、老之虛無,亦不取焦、京之術數。惟即辭占象變,敷陳人事,以明法天建極之實功。故御製序文,特揭以經學為治法之義焉。}

《御纂周易折中》二十二卷

\注{康熙五十四年,大學士李光地等奉敕撰。自董楷析朱子《周易本義》,附於《程傳》,十二篇舊第復淆。是編恭稟聖裁,改從古本,足正千古之訛。大旨雖根據程、朱,而參考群言,務求至當,實不偏主一家,允為說《易》之准繩。}

《御纂周易述義》十卷

\注{乾隆二十年,大學士傅恒等奉敕撰。以本「御纂周易折中」而推闡之,故名「述義」。大旨謂《易》因人事以立象,故不涉虛渺之說與術數之學。其觀象多取於互體,尤能發明古義。漢《易》、宋《易》,至是而集其成矣。}

《讀易大旨》五卷

\注{國朝孫奇逢撰。皆其讀《易》有得,錄示門人之語。其說不顯攻《圖》《書》,亦無一字及《圖》《書》,惟以象傳通一卦之旨,以一卦通六十四卦之義,皆切近人事,發明義理。末附「兼山堂問答」,及與李崶論《易》之語,別為一卷。崶即奇逢所從受《易》者也。}

《周易稗疏》四卷,附《考異》一卷

\注{國朝王夫之撰。皆隨筆札記,以剖析疑義。大旨不信焦、京,亦不信陳、邵,亦不取王弼之清言。惟引據訓詁,考求古義,所謂徵實之學也。}

《易酌》十四卷

\注{國朝刁包撰。大旨以程子《傳》、朱子《本義》為主,雖亦兼言象數,然皆陳摶、李之才之學,非漢以來之舊學也。取其持論篤實而已。}

《田間易學》十二卷

\注{國朝錢澄之撰。澄之初問《易》於黃道周,故頗詳於數學。後乃兼求義理,參取於王弼、孔穎達、程子、朱子之間。其謂《先天》《河》《洛》,皆因《易》而作圖,用錢義方之說;謂圖中奇偶乃揲蓍之法,非畫卦之本,用陳應潤之說也。}

《易學象數論》六卷

\注{國朝黃宗羲撰。宗羲以《易》至焦、京而流為方術,至陳摶而歧入道家,九流百氏,罔弗依託,因作此以糾其失。前三卷論《河圖》《洛書》、先天方位、納甲、納音、月建、卦氣、卦變、互卦、筮法、占法,附以所作原象,為內篇;後三卷論太玄、乾鑿度、元包、潛虛、洞極、洪範數、皇極數,以及六壬、太乙、遁甲,為外篇。}

《周易象辭》二十一卷,附《尋門餘論》二卷,《圖書辨惑》一卷

\注{國朝黃宗炎撰。宗炎力闢陳摶之學,故所解惟主義理,然根據經典,不涉空談。「尋門餘論」兼排釋氏,未免蔓衍於《易》外,而其他持論多醇正。「圖書辨惑」論《先天圖》,與陳應潤所言合;論《太極圖》,與朱彞尊、毛奇齡所考合。亦皆明確也。}

《周易筮述》八卷

\注{國朝王宏撰撰。以朱子謂《易》本卜筮之書,因作此編,以明其義。凡十五篇。雖端為揲蓍而作,然闢焦、京之小術,闡羲、文、周、孔之宏旨,立論悉本經義,與方技家所說迥殊,故進之列於《易》類,不以術數論焉。}

《仲氏易》三十卷

\注{國朝毛奇齡撰。是書述其兄錫齡之遺說,故以「仲氏」為名。大旨謂《易》兼五義,一曰變易,一曰交易,為先儒之所知;一曰反易,一曰對易,一曰移易,皆先儒之所未知。其言甚辨,然大致有所根據,非純構虛詞。}

《推易始末》四卷

\注{國朝毛奇齡撰。奇齡既作《仲氏易》,因採漢以來諸儒之言卦變者,別加綜核,以成是書。其名「推易」,蓋本《繫辭》剛柔相推之文,即《仲氏易》所謂「移易」也。}

《春秋占筮書》三卷

\注{國朝毛奇齡撰。摭《春秋傳》所載占筮,以明古人之《易》學。實為《易》作,非為《春秋》作也。}

《易小帖》五卷

\注{國朝毛奇齡說《易》之語,其門人記錄成書者也。凡一百四十三條,與《仲氏易》互相發明。大抵徵引古義,以糾近代說《易》之失。於王弼、陳摶二派,掊擊尤力。}

《喬氏易俟》十八卷

\注{國朝喬萊撰。前列諸圖,不取陳摶之說。於卦變亦不取虞翻諸家之說,而取來知德之反對。其解經多推求人事,證以史文,蓋李光、楊萬里之支流也。}

《讀易日抄》六卷

\注{國朝張烈撰。一以朱子《本義》為宗。因象設事,就事陳理,猶近時《易》說之不枝蔓者。}

《周易通論》四卷

\注{國朝李光地撰。一卷、二卷,發明上、下經大旨,三卷、四卷,發明繫辭、說卦、序卦、雜卦之義。冠以「易本」「易教」二篇,次論卦爻象彖、時位德應、《河圖》《洛書》,以及占筮掛扐、正變環互,皆一一詳悉。其於宋易,可謂融會貫通矣。}

《周易觀彖》十二卷

\注{國朝李光地撰。是書取《繫辭》「觀其彖辭,則思過半」之義,實注全經,非止解彖辭。其《語錄》《文集》,頗申明先天諸圖,此書則惟《說卦傳》「天地定位」一章,略及斯義,餘無一字及之,則亦知非畫卦之本矣。經中脫文誤字,惟《繫辭》「侯之」二字作衍文,餘皆不從《程傳》《本義》。其說皆自抒心得,亦不甚附合程、朱也。}

《周易淺述》八卷

\注{國朝陳夢雷撰。乃康熙甲戌夢雷戍尚陽堡時所作。大旨主《本義》,而參以王弼、孔穎達、蘇軾、來知德及永樂《大全》,蓋行篋乏書,故所據止此。其說多即象以明人事。末附三十圖,則其友楊道聲作也。}

《易原就正》十二卷

\注{國朝包儀撰。其學從《先天圖》入,故自序謂,《皇極經世》為《易》之本旨。然每爻注變卦,猶用古法,詮釋簡明,亦不繳繞奇偶,排比黑白,與自序實不相應也。}

《大易通解》十五卷

\注{國朝魏荔彤撰。其論畫卦,謂:《河圖》《洛書》,只可云其理相通,不必穿鑿附會。謂《先天圖》,非生卦之次序。論爻謂:當兼變爻。謂泰、否、損、益四卦,為上下經之樞紐,皆具有理解。惟不取扶陽抑陰之說,則未審姤、復之初爻矣。}

《易經衷論》二卷

\注{國朝張英撰。所釋惟六十四卦,每卦為論一篇。其立說主於顯易,不務艱深,頗能掃眾說之糾結。}

《易圖明辨》十卷

\注{國朝胡渭撰。其一卷辨《河圖》《洛書》,二卷辨五行九宮,三卷辨《參同契》《先天圖》《太極圖》,四卷辨《龍圖》《易數鉤隱圖》,五卷辨《啟蒙》《圖書》,六卷、七卷辨先天古易,八卷辨後天之學,九卷辨卦變,十卷辨象數流弊。並引據經典,元元本本,於《易》學深為有功。}

《合訂刪補大易集義粹言》八十卷

\注{國朝納喇性德撰。是書取宋陳友文《大易集義》、方聞一《大易粹貫》,案:原本訛「方聞一」為「曾穜」,今考正。刪除重複,刊削繁蕪,合為一編。宋儒易說,略具於斯。}

《周易傳註》七卷,附《周易筮考》一卷

\注{國朝李塨撰。其說以《易》卦本以人事立言,陳摶、劉牧諸圖,皆使《易》道入於無用;《參同契》《三易洞璣》之類,皆以異端方技闌入經學;即漢儒卦氣直日之類,亦經外別生枝節。故惟以觀象為主,第不廢互體耳。}

《周易札記》二卷

\注{國朝楊名時撰。名時《易》學,多得之其師李光地。是書惟《說卦傳》及附論《啟蒙》之類,頗推衍《先天》諸圖,餘皆發揮《易》理者也。}

《周易傳義合訂》十二卷

\注{國朝朱軾撰。凡《程傳》《本義》,互有異同者,務折中以歸一,使不涉兩岐。惟兩義並行不悖者,乃兼存其說,附以諸儒之論;或有實勝程、朱者,亦舍程、朱以從之。蓋不株守門戶之見也。}

《周易玩詞集解》十卷

\注{國朝查慎行撰。慎行受業於黃宗羲,故於《易》家一切雜學,灼然不惑。其河圖說、卦變說、天根月窟考、八卦相錯說、闢卦說、中爻互體說、廣八卦說,辨証具有根據。詮釋經文,亦明切不支。}

《惠氏易說》六卷

\注{國朝惠士奇撰。其書雜釋卦爻,專明漢學,大抵以象為主,而訓詁尤所加意。惟欲矯王弼等空言之弊,採掇未免駁雜,然其精核者,終不可廢也。}

《周易函書約存》十八卷,《約註》十八卷,《別集》十六卷

\注{國朝胡煦撰。原本一百十八卷,稿本浩繁,漸有散佚,其已刻者,亦編次無緒。此本乃其子季堂,以其論《易》之語,分為原圖、原卦、原爻、原占者,編為《約存》;以其依經釋義者,編為《約注》;而以「篝燈約旨」「易解辨異」「易學須知」,編為《別集》。其持論酌於漢學、宋學之間,與朱子頗有異同。}

《易箋》八卷

\注{國朝陳法撰。其書以《易》為專明人事,其駁來知德錯綜之說,最為明皙。其論筮法,亦具有理解。}

《楚蒙山房易經解》十六卷

\注{國朝晏斯盛撰,凡《易學初津》二卷,不取圖書之說,乃併卦變互體而廢之,未免主持稍過。《易翼宗》六卷,詮釋經文,附以《十翼》。《易翼說》八卷,詮釋《十翼》,又各自為篇,與何楷《古周易訂詁》例同,亦嫌繁複。然所解斟酌於言理、言數之間,則頗能持其平。}

《周易孔義集說》二十卷

\注{國朝沈起元撰。以孔子《十翼》為主,定眾說之是非。前列三圖:一曰八卦方位,一曰乾坤生六子,一曰因重。皆據《繫辭》《說卦》,其先天諸圖,則以為陳、邵之《易》,非孔子之《易》,概從芟剃,持論特確,所解亦多能推驗舊詁,引申新義。惟既用王弼散附之本,而又以大象、文言析出自為一傳,則自我作古耳。}

《易翼述信》十二卷

\注{國朝王又樸撰。其說亦以《十翼》為主。深以朱子所云「不可以孔子之《易》,為文王之《易》者」為非。其所徵引,惟李光地之說為多,亦不甚墨守《本義》也。}

《周易淺釋》四卷

\注{國朝潘思矩撰。大旨即象明理,而即互體卦變以求象,每卦皆註自某卦來,謂之「時來」;是亦《易》中之一義,不足盡《易》,而不可謂之非《易》,固可存備一家也。}

《周易洗心》九卷

\注{國朝任啟運撰。其說多發明圖學,謂《論語》之「五十學《易》」,即指《河圖》之五十,立論殊為新異。其詮釋經文,則觀象玩詞,時標精理;其考定文句,亦根據先儒。然則啟運之講圖學,特好語精微耳,非如張行成等,竟舍經而言數也。}

《豐川易說》十卷

\注{國朝王心敬撰。心敬所註諸經,皆好為異論。是書闡發《易》理,乃取諸人事,謂陰陽消長,不過借作影子,特為切實。惟排斥雜學,併《左傳》占法而詆之,為主持太過耳。}

《周易述》二十三卷

\注{國朝惠棟撰。其書主發揮漢儒之學,以荀爽、虞翻為主,而參以鄭玄、宋咸、干寶諸家之說,自為註而自疏之。凡二十一卷,中闕《下經》一卷,又闕序卦、雜卦傳,蓋未完之本。末二卷為《易微言》,雜抄經典論易之語,叢冗無緒,亦未及排纂之稿本也。}

《易漢學》八卷

\注{國朝惠棟撰。考漢《易》宗派源流,掇拾緒論,以見大凡。凡《孟長卿易》二卷;《虞仲翔易》一卷;《京君明易》二卷,干寶附焉;《鄭康成易》一卷;《荀慈明易》一卷。其末一卷,則棟發明漢《易》之理,以辨正《河圖》《洛書》、先天太極之學。}

《易例》二卷

\注{國朝惠棟撰。皆考究漢儒之傳,以發明《易》之本例。凡九十類。其中有錄無書者,十三類。所分門目,頗多牽混,蓋亦未成之稿。然棟於諸經,精研古義,其所採摭,多專門授受之學。儻因而排纂,猶可見作《易》之大綱,未可以冗雜棄也。}

《易象大意存解》一卷

\注{國朝任陳晉撰。多申尚象之旨,不載經文,惟折中諸家之說,明其大意。首論太極、五行、先天、河洛,皆鏟除;次論彖、論爻、論象;次論六十四卦,多指陳法戒;終以系辭、說卦、序卦、雜卦,其文頗略,以所重在六十四卦也。}

《大易擇言》三十六卷

\注{國朝程廷祚撰。因桐城方苞《緒論》,以六例編纂諸家之說。一曰正義,二曰辨正,三曰通論,四曰餘論,五曰存疑,六曰存異。大抵力排象數,惟以義理為宗。}

《周易辨畫》四十卷

\注{國朝連斗山撰。大旨謂一卦之義,在於爻;爻畫有剛、有柔,因剛柔之畫而立之象,即因剛柔之畫而繫以辭,其道先在於辨畫,故以為名。雖不免或涉穿鑿,然逐卦剖析互體,亦時有精理。}

《周易圖書質疑》二十四卷

\注{國朝趙繼序撰。其書以象數言《易》,而不主先天河洛之說。首為古經十二篇,次逐節詮釋經義,而不載經文,蓋用經傳別行之古例。次為圖三十有二,為說五。其詁經多從卦變起象,而兼取漢、宋之說,無所偏主。}

《周易章句證異》十一卷

\注{國朝翟均廉撰。皆考究諸本,辨《周易》篇章字句之同異,校勘頗為精密。附錄}

《乾坤鑿度》二卷

\注{是書為《永樂大典》所載《易緯》八種之一。分上、下二篇。上篇論四門四正取象取物,以至卦爻蓍策之數;下篇論坤有十性,而推及於蕩配凌配。又雜引諸緯書之詞,佶屈聱牙,頗不易曉。}

《周易乾鑿度》二卷

\注{是書為《易緯》八種之二。舊本標鄭康成注。唐以前說經之家,恒相引用。其太乙行九宮法,即後世《洛書》所從出。在緯書之中,特為醇正。}

《易緯稽覽圖》二卷

\注{是書為《易緯》八種之三。首言卦氣起中孚,而以坎、離、震、兌為四正卦。六十卦,卦主六日七分。又以自坤至複十二卦為消息。餘雜卦,主公卿侯大夫,候風雨寒溫以為征。應即孟喜、京房之學。至所稱軌之數,以及世應游歸,乃兼通日家推步之法。唐一行《大衍歷議》,即演其術。惟所紀年號至唐元和,疑術家所附益也。}

《易緯辨終備》一卷

\注{是書為《易緯》八種之四。一作「辨中備」,傳寫異文也。今《永樂大典》所載,僅數十言,似非完本。以古來著錄,姑存以備考核。}

《易緯通卦驗》二卷

\注{是書為《易緯》八種之五。《宋史·藝文志》作二卷,《永樂大典》所載合為一篇。今核其文義,定「人主動而得天地之道,則萬物之精盡矣」以上為上卷。曰「凡易八卦之氣,驗應各如其法度」以下為下卷。上言稽應之理,下言卦氣之徵驗也。}

《易緯乾元序制記》一卷

\注{是書為《易緯》八種之六。唐以前史不著錄,陳振孫《書錄解題》始載之。然其文乃與諸書所引《是類謀》《坤靈圖》《稽覽圖》之文相同,疑後人割裂緯書,偽題此名也。}

《易緯是類謀》一卷

\注{是書為《易緯》八種之七。通體以韻語成文,多言禨祥推驗,並及於姓輔名號,與《乾鑿度》所引易歷義相發明。}

《易緯坤靈圖》一卷

\注{是書為《易緯》八種之八。殘缺不完,僅存論乾、大蓄、無妄卦辭,及史注所引「日月合璧」數語而已。}

\begin{段落}[indent=1]
右易類。共一百五十八部,一千七百五十七卷。附錄八部,十二卷。
\end{段落}
\end{正文}
\end{document}
