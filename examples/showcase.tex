\documentclass[四库全书]{ltc-guji}
\title{欽定四庫全書}
\chapter{功能展示}

\begin{document}
\begin{正文}

功能展示

\begin{列表}
    \item 竖排排版
    \item 夹注与侧批
    \item 装饰与边框
\end{列表}

基本竖排:天地玄黃宇宙洪荒日月盈昃辰宿列張\\
寒來暑往秋收冬藏閏餘成歲律呂調陽

夹注示例:黄帝者\夹注{集解徐廣曰號有熊索隠按有土徳之瑞土色黄故稱黄帝}少典之子\夹注{集解譙周曰有熊國君少典之子也皇甫謐曰有熊今河南新鄭是也}姓公孫名曰軒轅

\begin{段落}[indent=2]
段落缩进两格的效果\\
第二列也缩进
\end{段落}

侧批:天子\侧批{此处为侧批}诸侯\侧批{又一侧批}卿大夫

装饰功能:
\圈点{重要文字}需要圈点标注
\专名号{司馬遷}为史记作者
\书名号{史記}是纪传体通史

特殊文本框:
\反白{卷一}为反白效果
\带圈{壹}为带圈效果

空格\空格[3]控制

\批注[x=3cm, y=1cm]{页面任意位置的浮动批注}

\end{正文}
\end{document}
